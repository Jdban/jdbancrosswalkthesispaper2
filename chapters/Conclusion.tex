\chapter{Conclusion}
\label{conclusion}
%What I did ( couple sentences), why I did it (more), why I think they worked or didn't work  (biggest)
%more about future uses and less about results

This project incorporated neural networks into a process of recognizing zebra stripe crosswalks in images. This process may be effective in aiding a visually impaired user across a crosswalk. 

Neural networks are used in many image processing platforms, but they haven't yet been applied to recognizing zebra crosswalks in this way.

%An improvement of about 887\% increase in q-value was seen when comparing the same code using neural networks vs not using neural networks on the same dataset. With neural networks, there was an increase in runtime due to the calculations required for the neural network, but with optimizations that could be reduced. The dataset used was reasonably large and diverse and garnered decent results. It would definitely be worth looking into this method further for real world use. 

Using the same dataset, a substantial improvement in false discovery rate (937\%) was seen when comparing the same code using neural networks versus without them. The false discovery rate is important because the higher it is, the more confidence can be given to a positive prediction. There was an increase in runtime due to the calculations required for the neural network inputs, but optimizations could reduce that impact. The dataset used was large and diverse. The results presented in this paper show that using neural networks is a promising tool in zebra crosswalk detection, and is well worth further investigation.