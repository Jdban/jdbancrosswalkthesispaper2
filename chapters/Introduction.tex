\chapter{Introduction}
\label{intro}

Crossing the street can be dangerous for visually impaired individuals because orienting oneself is difficult without being able to use a visual marker. Visually impaired people must use non-visual cues such as listening for cars stopped at red lights and trying to walk in front of them. Vision processing could be applied to recognize the crosswalk to assist the user with limited vision. In order to increase the safety of blindly crossing the street, a phone app could allow the user to hold up their phone and get feedback about the location of the crosswalk. As they cross the street, they would receive feedback as to whether they were going off course. Developing a fast and reliable metric for detecting the crosswalk is the first step towards developing a vision processing phone app to aid the visually impaired.

There have been some investigations of zebra crosswalk detection, but none so far have used neural networks \cite{Coughlan2006}\cite{ZebraPhone}\cite{relatedworkbipolarity}. Neural networks are formed by a number of interconnected nodes that require training. Once trained, they are configured to a specific application to predict outcomes based on the inputs. Neural networks are being used for an increasing number of applications where patterns are difficult to decipher manually. Although the training process can be time consuming, neural networks run quickly once trained, which is a large benefit to any application where speed is a factor. 

Figure-ground segmentation has been used to separate the objects from the background scene. This technique can be applied to zebra crosswalks to allow the detection of potential stripelets in the image using geometric parameters \cite{Coughlan2006}. Adding neural networks to this discovery method allows us to use parameters to evaluate whether stripelets are part of a crosswalk, a task which would be difficult manually. This paper aims to use neural networks to improve the reliability of zebra crosswalk detection methods, by offering a different approach to the problem.